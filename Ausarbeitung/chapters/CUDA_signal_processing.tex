% !TeX spellcheck = de_DE

\section[Teilprojekt: \enquote{Parallele Fourieranalyse mit CUDA}]{Teilprojekt: \enquote{Parallele Fourieranalyse mit CUDA}} \label{CUDA_DSP}

\subsection{Überblick}
Um ein Gesamtbild über das implementierte Projekt zu bekommen, wird an dieser Stelle ein grober Rahmen, der die wichtigsten Funktionalitäten und Anforderungen an die Implementierung umfasst, angegeben.
Aus dem Abschnitt Idee und Zielsetzung \ref{idea_goal} ist bereits bekannt, dass Audiodaten in Form von WAV-Dateien analysiert und deren anteilige Frequenzen dargestellt werden, was in einer sequentiellen Umgebung zunächst kein größeres Problem darstellt.
Diese Arbeit geht jedoch einen Schritt weiter, parallelisiert diesen Prozess und nutzt die Schnittstelle CUDA aus, um je nach verwendeter Hardware dieses Problem auf die Grafikkarte zu skalieren. 
Mit dem Schritt der Parallelisierung stellen sich folgende Fragen:

\begin{itemize}
	\item Wie werden die Audiodaten auf die Shader-Kerne der Grafikkarte übertragen?
	\item Welche Form haben diese Daten?
	\item Wie und in welchen Intervallen werden diese Daten portioniert und wie viele Daten werden parallel verarbeitet?
	\item Mit welcher Methode werden die Daten verarbeitet bzw. umgerechnet?
	\item Wie werden die Daten nach erfolgreicher Berechnung wieder verfügbar gemacht?
	\item Form der Visualisierung?
\end{itemize}

Einige dieser Fragen bringen komplexere Strukturen mit sich und sind zunächst nicht einfach zu beantworten.
Eine Audiodatei ist im Grunde genommen eine Textdatei, mit vielen Hexadezimalen Werten und einer Header-Information, wie in Abschnitt WAV-Format \ref{sub:WAV} beschrieben wurde.
Aus dieser Datei muss die Information über Sample-Rate, Anzahl der Kanäle und Größe eines Samples extrahiert werden. Dies wird im weiteren Verlauf für die Aufteilung der Samples auf die CUDA-Kerne und Bestimmung der Frequenzen verwendet.
Die Sampledaten werden mithilfe des \enquote{StreamWriter}-Subprojekts über eine TCP-Schnittstelle verfügbar gemacht. Diese TCP-Schnittstelle wird von einem weiteren Teilprojekt AudioTracer benutzt, um die Daten weiter zu verarbeiten.
Das Modul \texttt{AudioTracer} liegt in zweifacher Ausführung in den Programmiersprachen C und Python vor. Beide Varianten liefern dieselbe Ausgabe und können somit auf der Ebene von Performance und Semantik verglichen werden. 
Die Aufgabe der Implementierung ist, die über die TCP-Schnittstelle empfangenen Audio Sampledaten in eine für CUDA passende Form zu bringen und diese mithilfe von einigen API-Anweisungen und Verteilungsentscheidungen auf die CUDA-Kerne der GPU zu partitionieren.
Jeder dieser CUDA-Kerne führt dann eine Fouriertransformation auf den jeweils zugesicherten Teilbereich durch. Da die Audiodaten sehr viel mehr Samples beinhalten, als CUDA-Kerne vorhanden sind, muss vorher entschieden werden, wie viele Samples auf einmal bearbeitet werden können.

\subsection{StreamWriter(C/C++)}

Die Aufgabe des Teilprojekt StreamWriter ist es, Audiodateien einzulesen und diese Daten über eine Schnittstelle für andere Teilprojekte verfügbar zu machen.
Um die Header-Informationen zu extrahieren, wird die Datei im Binärmodus geöffnet. Bestimmte Bereiche des Headers sind mit verschiedenen Größen versehen. 
Die Verwendung eines \enquote{typedef struct} ermöglicht es, die Größen der Variablen vordefinieren zu können. 
Im weiteren Verlauf werden alle Headerinformationen durch diese Technik herausgefiltert.
Die wichtigen bzw. in diesem Projekt verwendeten Headerinformationen sind Anzahl der Audiokanäle (NumChannels), Sample Rate (in der Wiedergabe abgespielte Audiosamples pro Sekunde), Bitanzahl pro Sample (BitsPerSample), sowie die eigentlichen Samples.
Die Daten liegen in Hexadezimalform im Little-Endian Format vor. Diese Hexadezimalen Werte werden im Integer Format gespeichert. 
Der hier beschriebene Teil des Programms \enquote{StreamWriter} ist in der Datei WAV.c bzw. WAV.h implementiert. 
Die Implementierung wird in allen weiteren Varianten verwendet, um eine WAV-Datei einzulesen und als Integer-Array speichern zu können.
Analog zur Zielsetzung, werden die gespeicherten Audiodaten mithilfe einer Schnittstelle für das AudioTracer Programm zur Verfügung gestellt.
Im Laufe der Bearbeitung dieser Arbeit sind dabei zwei Generationen von Schnittstellen entstanden, auf die im Folgenden Bezug genommen wird.
Die grundsätzliche Idee war es, die Sampledaten dem AudioTracer als kontinuierlichen Datenstrom zur Verfügung zu stellen. 
Die erste Version erzeugt eine Datei, in der die Daten der Samples als Integer-Werte geschrieben wurden. Diese wurden an dieser Stelle im AudioTracer wieder stückweise abhängig eines definierten Threshholds eingelesen.
Die zweite Generation dieser Schnittstelle sieht vor, die Daten über das Netzwerkprotokoll TCP zu versenden. Das AudioTracer-Modul greift diese Verbindung auf der anderen Seite auf und liest die Daten in vordefinierten Größen ein.

\paragraph{Datei-Variante}

In dieser Generation der Datenschnittstelle werden die Audiodaten über eine \enquote{Vermittlerdatei} an das AudioTracer-Projekt weitergeleitet. 
Die zuvor in Integer umgewandelten Werte werden in hexadezimaler Form in die Datei \enquote{out.dat} geschrieben.
Jeweils 1000 Werte je Audiokanal werden pro Iteration der Routine in dieser Datei hinzugefügt.
An dieser Stelle entsteht folgendes Problem: Der Zugriff auf eine Datei erfolgt sequentiell, d.h., es ist nur jeweils ein Zugriff zur gleichen Zeit auf eine Datei systembedingt gestattet.
Das AudioTracer-Modul verlangt lesenden Zugriff auf die zuvor geschriebenen Daten.
Im StreamWriter-Modul muss die Datei somit zwangsweise geschlossen werden, damit ein weiterer Zugriff seitens des AudioTracers stattfinden kann. 
Hier kann es unter Umständen zu einem Starvation-Problem kommen, wodurch das AudioTracer Modul keinen lesenden Zugriff auf die Datei erlangen kann.
Im StreamWriter muss daher eine \enquote{Zwangspause} in der Schleifeniteration implementiert werden. Um beiden Modulen einen Zugriff zu ermöglichen, wird diese Pause durch \texttt{Sleep(rand() \% 50)} realisiert.
Eine Randomisierung löst das Problem, dass auch bei festen Wartezeiten eine solche Starvation-Situation entstehen kann, wenn z.B. die Ausführungszeiten des AudioTracer-Moduls immer leicht über der vordefinierten Pause liegen.

\paragraph{TCP-Variante}

Diese Generation der Schnittstelle wird anstelle einer Datei, die als Vermittler dient, durch das TCP-Netzwerkprotokoll realisiert. 
Diese Variante des \\StreamWriter-Moduls wartet auf eine erfolgreiche Verbindung auf der anderen Seite der TCP-Verbindung. 
Sobald diese Verbindung besteht, werden die Integer Daten in einen String von Hexadezimalwerten umgewandelt. Die Umwandlung in Hexadezimal erfolgt, um eine gleichbleibende Größe der TCP-Pakete zu gewährleisten.
Bei einem 16Bit-PCM WAV-Format ist ein Sample in einem Kanal jeweils zwei Byte groß, sodass vier Byte pro Zeitpunkt an Speicher benötigt werden. 
Die hexadezimalen Werte lassen sich leicht durch die Verwendung eines \texttt{stringstream}-Objekts in ein String Objekt umwandeln, welches für das Senden von Daten benötigt wird. 
Wie im vorherigen Abschnitt beschrieben, werden pro Iteration 1000 Samples pro Audiokanal gesendet und somit beträgt die Größe der versendeten Strings bei der TCP-Variante 10000 Bytes, da die Audiokanäle durch Leerzeichen und die einzelnen Sample-Gruppen durch ein \texttt{\textbackslash n} Steuerzeichen abgegrenzt sind.
Eine \enquote{Zwangspause} wie in der ersten Generation ist hier durch die Verwendung des TCP-Protokolls nicht nötig, da eine direkte Pipeline zwischen AudioTracer- und StreamWriter-Modul besteht.

