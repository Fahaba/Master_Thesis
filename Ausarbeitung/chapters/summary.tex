% !TeX spellcheck = de_DE

\section{Zusammenfassung}

In dieser Arbeit wurden zwei Projekte vorgestellt, wobei das erstere AudioTracer Projekt größere Entwicklungs- und Recherchearbeit erfordere. Dieses besteht aus mehreren Modulen, die eigene Zuständigkeitsbereiche abdecken. Analog zu Abbildung \ref{fig:OV} besteht dieses Teilprojekt aus den Modulen StreamWriter, jeweils einem CUDA Modul für C und Python und aus einem Bokeh Webserver Modul. Der StreamWriter liest eine WAV-Datei ein und macht die darauf gewonnenen Sampledaten für das C- oder Python- CUDA Modul in Form eines TCP-Streams verfügbar. Das C CUDA Modul empfängt diese Daten bis zu jeweils einem gewissen Schwellwert, wandelt diese in eine komplexe Repräsentation und startet eine CUDA Routine. Jedem CUDA Kernel Thread wird ein Teilbereich der Daten zugewiesen. Die genaue Aufteilung kann der Sektion \ref{sec:audiotracer} entnommen werden. Jeder CUDA Thread wendet auf die zugewiesenen Teilbereiche eine Fourier Transformation an. Im Falle des C-Moduls wird eine diskrete Fouriertransformation verwendet. Die Ausgabe beschreibt, welche harmonischen Frequenzen wie stark in den übermittelten Audiodaten anteilig sind.

Das Python CUDA Modul weist einen ähnlichen Aufbau zum C-Modul auf. Prägnant ist jedoch, dass der Quellcode im Vergleich durch weniger notwendige Umformungen und Typtreue - wahrscheinlich mit Performanceverlusten - wesentlich kleiner ausfällt. Ein Performance Vergleich wurde hier nicht vorgenommen. Das Python Modul benutzt eine Fast Fourier Transformation aus dem Paket \texttt{cupy} um die Frequenzen zu extrahieren.

Beide CUDA Module kommunizieren das Ergebnis Array über eine weitere TCP-Schnittstelle mit dem Python Bokeh Server Modul. Dabei handelt es sich um einen Visualisierungswebserver, der mithilfe von AJAX POST-Requests Daten an den Browser weiterleiten kann, die durch vorgefertigte JavaScript-Anbindungen in eine SVG-Grafik visualisiert werden.

Der Ablauf von StreamWriter bis zum Browser findet im Kontext zu Grafik \ref{fig:OV} von links nach rechts statt.

Das zweite Teilprojekt AudioParkour bietet eine Möglichkeit, eine WAV-Datei während der laufenden Wiedergabe über CUDA Routinen zu manipulieren, um den Effekt einer positionsbezogenen Audiowiedergabe zu simulieren. Dieses Projekt wurde in Python implementiert. Python liefert die notwendigen Pakete \texttt{Sounddevice} und \texttt{PyCUDA}. Ersteres ermöglicht eine Implementierung eines Audio Streams, wobei immer eine bestimmte Anzahl an Audiosamples wiedergegeben werden, bis eine Callback Funktion aufgerufen wird, mit der die Audiodaten des nächsten Blocks festgelegt werden können. Parallel dazu können Nutzereingaben getätigt werden, die bewirken, dass CUDA Operationen vorgenommen werden, die die Signaldaten in Abhängigkeit der getätigten Nutzereingaben manipuliert. Diese daraus gewonnen Daten werden in der Callback Methode zu gegebenem Zeitpunkt verwendet. 