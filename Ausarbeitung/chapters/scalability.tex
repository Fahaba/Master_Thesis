\section{Skalierbarkeit}

Die beiden Teilprojekte AudioTracer und Audioparcours wurden mit verschiedenen Testeingaben ausgeführt und entsprechende Ergebnisse bereits diskutiert. An diesem Punkt bleibt die Frage der Skalierbarkeit beider Teilprojekte unbeantwortet. 

Das \textbf{AudioTracer} Projekt wurde analog zu Kapitel \ref{audiotracer_results} in acht Testfällen getestet, jeweils vier für das C- bzw. Python. Die Anzahl der Samples und Audiokanäle variierten dabei zwischen 5000 und 25000 bzw. zwei und vier. Bei diesen Testmengen werden die CUDA-Kernels nur wenig beansprucht. Außerdem sind die Ergebnisse aus genannten Gründen teilweise ungenau. Die Anzahl von Samples und Audiokanälen kann durch Veränderung von Variablen vorgenommen werden. Diese müssen in den Modulen StreamWriter, AudioTracer und Bokeh Server verändert werden. Die Aufgaben der CUDA-Kernel werden generisch anhand dieser Parameter aufgeteilt, wodurch die Frage nach der Skalierbarkeit im Falle des AudioTracer-Moduls beantwortet werden kann. Ein sinnvoller nächster Schritt für die Erweiterung der Testreihen wären acht Audiokanäle mit einem Threshhold von 160000 Samples. Um die maximale Skalierbarkeit bei größtmöglicher Genauigkeit zu testen ist es wichtig, dass die Erhöhung der Audiokanäle und Sample-Anzahl im besten Fall so gewählt wird, dass die Eingabemenge für der Fouriertransformation nach der Formel $\#channels * 20000$ gewählt wird. Die Genauigkeit würde dann bei 1Hz liegen. Die Grenzen dieses Moduls sind in der Theorie schwierig zu bestimmen, kann jedoch mit dem oben genannten Ansatz schnell bestimmt werden. Die Probleme werden vermutlich einen Bottleneck beim Empfangen der Daten über die TCP-Schnittstelle aufweisen. Hier könnten Verbesserungen erzielt werden. Im Kapitel \ref{chap:farsight} werden kurz die Vorteile der CUDA-Streams erwähnt, wodurch die Ausführung des Programms weiterläuft, während die CUDA-Aufgaben in einer Warteschlange eingereiht werden. Dadurch wäre das Warten auf die Verfügbarkeit der Daten weniger präsent.

Die Skalierbarkeit für das \textbf{AudioParcours} Teilprojekt im Vergleich schwieriger zu bestimmen. Das Projekt ist für die Manipulation einer Audioquelle ausgelegt. Eine Erweiterung des Quellcodes würde vorsehen, mehrere Audiostreams vom Paket Sounddevice zu erstellen. Sollte die Möglichkeit bestehen den jeweiligen Audiostream zu identifizieren, würde \underline{ein} Callback reichen, um alle Streams in Abhängigkeit der Nutzereingaben zu manipulieren. Ansonsten würde der Quellcode aufgebläht, da für jeden Stream ein jeweiliger Callback vorhanden sein müsste. Außerdem muss im Voraus bekannt sein, an welcher Stelle sich die Audioquellen befinden. Die Klasse \enquote{Position} müsste im Zusammenhang jedes Streams verwendet werden, da sich die Ausgangswerte unterscheiden. Dadurch wäre eine Untersuchung der Skalierbarkeit möglich. Ein anderer Ansatz wäre die Annahme, dass nicht die Anzahl der Audioquellen steigt, sondern die der Audiokanäle. Wären die Kanäle an festen Positionen \enquote{stationiert}, wäre es möglich, bei steigender Anzahl der Kanäle die Skalierung des Audioparcours zu untersuchen. Voraussichtlich würden jedoch ab einer gewissen Anzahl von Tonobjekten bzw. Audiokanälen eine Latenz zwischen Nutzereingaben und hörbarer Veränderung in der Wiedergabe entstehen. 